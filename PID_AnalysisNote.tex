\documentclass[a4paper,10pt,notitlepage]{report}
\usepackage[utf8]{inputenc}
\usepackage{authblk}
\usepackage{geometry}
\usepackage{graphicx}
\usepackage{float}
\usepackage{cite}
\usepackage{caption}
\usepackage{subcaption}
\usepackage{hyperref}
\usepackage[numbers]{natbib}
\usepackage{rotating}
\usepackage{lineno}
\usepackage{amsmath}
\usepackage{amssymb}

\makeatletter
\setlength{\@fptop}{0pt}
\makeatother

\pdfinfo{%
  /Title    ()
  /Author   (Lukasz Fulek)
}




%chapter heading
\makeatletter
\renewcommand{\@makechapterhead}[1]{%
 \vspace*{18\p@}%
  {\parindent \z@ \raggedright
%     \LARGE \bfseries \thechapter. #1\par\nobreak
%     \vskip 40\p@
      \Huge \bfseries \thechapter. #1\par\nobreak
      \vskip 40\p@
  }}
\makeatother


% Title Page
\title{\textbf{Measurement of particle production\\with Roman Pot detectors in diffractive proton-proton interactions at~$\sqrt{s}=$~200~GeV}\vspace*{10pt}}
\author[1]{Leszek Adamczyk}
\author[1]{Łukasz Fulek}
\affil[1]{AGH University of Science and Technology, Kraków, Poland}

\setcounter{Maxaffil}{0}
\renewcommand\Affilfont{\itshape\small}
\renewcommand{\bibname}{References}

\begin{document}

\begin{center}
\begin{minipage}[c]{0.12\linewidth}%
\vspace{5.5pt}\textbf{\LARGE{of the}}
\end{minipage}
\begin{minipage}[c]{0.15\linewidth}%
\hspace*{-8pt}\includegraphics[width=\linewidth]{graphics/STAR_logo.pdf}
\end{minipage}~
\begin{minipage}[c]{0.24\linewidth}%
\vspace{9pt}\hspace*{-8pt}\textbf{\LARGE{Experiment}}
\end{minipage}\\[-50pt]
\textbf{\LARGE{Analysis Note}}

\vspace*{150pt}
\begin{minipage}{\linewidth}
\maketitle
\begin{abstract}
In this note we present the analysis of inclusive diffraction, focusing on the spectra of identified charged particles as pions, kaons and protons and their anti-particle counterparts in Single Diffraction $\left(p+p\to p+X\right)$ and Central Diffraction $\left(p+p\to p+X+p \right)$ processes.  Moreover, the $\bar{p}/p$ ratio as a function of rapidity is presented
to study the baryon number transfer from forward to midrapidity
in Single Diffraction. Similar effect has been studied in proton-proton and proton-photon interactions but it is the first measurement in proton-Pomeron interaction.
This data come from proton-proton collisions collected in 2015
The forward proton(s) were tagged in the STAR Roman Pot system while the identified charged particle tracks were reconstructed in the~STAR Time Projection Chamber (TPC). 
Ionization energy loss and time-of-flight of charged particles were used for particle identification. 

We describe all stages of the analysis involving the extraction of efficiency and acceptance corrections, comparison of the data with MC simulations and systematic uncertainty studies.	
	

\end{abstract}
\thispagestyle{empty}
\end{minipage}

\vspace{100pt}

 \Huge{\textbf{\textit{DRAFT}}}
\end{center}


\clearpage
\thispagestyle{empty}
\newgeometry{hmargin={2cm, 2cm}, height=10.0in}
\tableofcontents



%% =====  DATASET ====
\input{Introduction.tex}
%\input{ExperimentalSetup.tex}
\input{Dataset.tex}
\input{Analysis.tex}
\chapter{Acceptance and Efficiency}\label{chap:AcceptanceEfficiency}
\input{SystematicStudy.tex}
\input{Results.tex}


% %% ===== DODATKI ===== ------------
% \begin{appendices}
% \input{Appendix_RunList.tex}
% \input{Appendix_tDistributions.tex}
% \input{Appendix_AngularBeamDivergence.tex}
% \input{Appendix_BunchProfiles.tex}
% \end{appendices}


\addcontentsline{toc}{chapter}{References}
\bibliography{references.bib}{}
\bibliographystyle{utphys}

\end{document}          
